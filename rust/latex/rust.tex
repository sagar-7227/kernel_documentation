%% Generated by Sphinx.
\def\sphinxdocclass{report}
\documentclass[a4paper,11pt,english]{sphinxmanual}
\ifdefined\pdfpxdimen
   \let\sphinxpxdimen\pdfpxdimen\else\newdimen\sphinxpxdimen
\fi \sphinxpxdimen=.75bp\relax

\PassOptionsToPackage{warn}{textcomp}


\usepackage{cmap}
\usepackage[T1]{fontenc}
\usepackage{amsmath,amssymb,amstext}
\usepackage{babel}



\usepackage{times}
\expandafter\ifx\csname T@LGR\endcsname\relax
\else
% LGR was declared as font encoding
  \substitutefont{LGR}{\rmdefault}{cmr}
  \substitutefont{LGR}{\sfdefault}{cmss}
  \substitutefont{LGR}{\ttdefault}{cmtt}
\fi
\expandafter\ifx\csname T@X2\endcsname\relax
  \expandafter\ifx\csname T@T2A\endcsname\relax
  \else
  % T2A was declared as font encoding
    \substitutefont{T2A}{\rmdefault}{cmr}
    \substitutefont{T2A}{\sfdefault}{cmss}
    \substitutefont{T2A}{\ttdefault}{cmtt}
  \fi
\else
% X2 was declared as font encoding
  \substitutefont{X2}{\rmdefault}{cmr}
  \substitutefont{X2}{\sfdefault}{cmss}
  \substitutefont{X2}{\ttdefault}{cmtt}
\fi


\usepackage[Bjarne]{fncychap}
\usepackage{sphinx}
\sphinxsetup{
        hmargin=0.5in, vmargin=1in,
        parsedliteralwraps=true,
        verbatimhintsturnover=false,
    }
\fvset{fontsize=\small}
\usepackage{geometry}
\usepackage{setspace}

% Include hyperref last.
\usepackage{hyperref}
% Fix anchor placement for figures with captions.
\usepackage{hypcap}% it must be loaded after hyperref.
% Set up styles of URL: it should be placed after hyperref.
\urlstyle{same}

\usepackage{sphinxmessages}
\setcounter{tocdepth}{0}


        % Use some font with UTF-8 support with XeLaTeX
        \usepackage{fontspec}
        \setsansfont{DejaVu Sans}
        \setromanfont{DejaVu Serif}
        \setmonofont{DejaVu Sans Mono}
    
        % Load kerneldoc specific LaTeX settings
	\input{kerneldoc-preamble.sty}


\title{Linux Rust Documentation}
\date{Aug 21, 2023}
\release{5.4.10-custom}
\author{The kernel development community}
\newcommand{\sphinxlogo}{\vbox{}}
\renewcommand{\releasename}{Release}
\makeindex
\begin{document}

\ifdefined\shorthandoff
  \ifnum\catcode`\=\string=\active\shorthandoff{=}\fi
  \ifnum\catcode`\"=\active\shorthandoff{"}\fi
\fi

\pagestyle{empty}
\sphinxmaketitle
\pagestyle{plain}
\sphinxtableofcontents
\pagestyle{normal}
\phantomsection\label{\detokenize{index::doc}}


Documentation related to Rust within the kernel. To start using Rust
in the kernel, please read the {\hyperref[\detokenize{quick-start::doc}]{\sphinxcrossref{\DUrole{doc}{Quick Start}}}} guide.


\chapter{Quick Start}
\label{\detokenize{quick-start:quick-start}}\label{\detokenize{quick-start::doc}}
This document describes how to get started with kernel development in Rust.


\section{Requirements: Building}
\label{\detokenize{quick-start:requirements-building}}
This section explains how to fetch the tools needed for building.

Some of these requirements might be available from Linux distributions
under names like \sphinxcode{\sphinxupquote{rustc}}, \sphinxcode{\sphinxupquote{rust\sphinxhyphen{}src}}, \sphinxcode{\sphinxupquote{rust\sphinxhyphen{}bindgen}}, etc. However,
at the time of writing, they are likely not to be recent enough unless
the distribution tracks the latest releases.

To easily check whether the requirements are met, the following target
can be used:

\begin{sphinxVerbatim}[commandchars=\\\{\}]
make LLVM=1 rustavailable
\end{sphinxVerbatim}

This triggers the same logic used by Kconfig to determine whether
\sphinxcode{\sphinxupquote{RUST\_IS\_AVAILABLE}} should be enabled; but it also explains why not
if that is the case.


\subsection{rustc}
\label{\detokenize{quick-start:rustc}}
A particular version of the Rust compiler is required. Newer versions may or
may not work because, for the moment, the kernel depends on some unstable
Rust features.

If \sphinxcode{\sphinxupquote{rustup}} is being used, enter the checked out source code directory
and run:

\begin{sphinxVerbatim}[commandchars=\\\{\}]
rustup override set \PYGZdl{}(scripts/min\PYGZhy{}tool\PYGZhy{}version.sh rustc)
\end{sphinxVerbatim}

Otherwise, fetch a standalone installer or install \sphinxcode{\sphinxupquote{rustup}} from:
\begin{quote}

\sphinxurl{https://www.rust-lang.org}
\end{quote}


\subsection{Rust standard library source}
\label{\detokenize{quick-start:rust-standard-library-source}}
The Rust standard library source is required because the build system will
cross\sphinxhyphen{}compile \sphinxcode{\sphinxupquote{core}} and \sphinxcode{\sphinxupquote{alloc}}.

If \sphinxcode{\sphinxupquote{rustup}} is being used, run:

\begin{sphinxVerbatim}[commandchars=\\\{\}]
rustup component add rust\PYGZhy{}src
\end{sphinxVerbatim}

The components are installed per toolchain, thus upgrading the Rust compiler
version later on requires re\sphinxhyphen{}adding the component.

Otherwise, if a standalone installer is used, the Rust repository may be cloned
into the installation folder of the toolchain:

\begin{sphinxVerbatim}[commandchars=\\\{\}]
git clone \PYGZhy{}\PYGZhy{}recurse\PYGZhy{}submodules \PYGZbs{}
        \PYGZhy{}\PYGZhy{}branch \PYGZdl{}(scripts/min\PYGZhy{}tool\PYGZhy{}version.sh rustc) \PYGZbs{}
        https://github.com/rust\PYGZhy{}lang/rust \PYGZbs{}
        \PYGZdl{}(rustc \PYGZhy{}\PYGZhy{}print sysroot)/lib/rustlib/src/rust
\end{sphinxVerbatim}

In this case, upgrading the Rust compiler version later on requires manually
updating this clone.


\subsection{libclang}
\label{\detokenize{quick-start:libclang}}
\sphinxcode{\sphinxupquote{libclang}} (part of LLVM) is used by \sphinxcode{\sphinxupquote{bindgen}} to understand the C code
in the kernel, which means LLVM needs to be installed; like when the kernel
is compiled with \sphinxcode{\sphinxupquote{CC=clang}} or \sphinxcode{\sphinxupquote{LLVM=1}}.

Linux distributions are likely to have a suitable one available, so it is
best to check that first.

There are also some binaries for several systems and architectures uploaded at:
\begin{quote}

\sphinxurl{https://releases.llvm.org/download.html}
\end{quote}

Otherwise, building LLVM takes quite a while, but it is not a complex process:
\begin{quote}

\sphinxurl{https://llvm.org/docs/GettingStarted.html\#getting-the-source-code-and-building-llvm}
\end{quote}

Please see Documentation/kbuild/llvm.rst for more information and further ways
to fetch pre\sphinxhyphen{}built releases and distribution packages.


\subsection{bindgen}
\label{\detokenize{quick-start:bindgen}}
The bindings to the C side of the kernel are generated at build time using
the \sphinxcode{\sphinxupquote{bindgen}} tool. A particular version is required.

Install it via (note that this will download and build the tool from source):

\begin{sphinxVerbatim}[commandchars=\\\{\}]
cargo install \PYGZhy{}\PYGZhy{}locked \PYGZhy{}\PYGZhy{}version \PYGZdl{}(scripts/min\PYGZhy{}tool\PYGZhy{}version.sh bindgen) bindgen
\end{sphinxVerbatim}


\section{Requirements: Developing}
\label{\detokenize{quick-start:requirements-developing}}
This section explains how to fetch the tools needed for developing. That is,
they are not needed when just building the kernel.


\subsection{rustfmt}
\label{\detokenize{quick-start:rustfmt}}
The \sphinxcode{\sphinxupquote{rustfmt}} tool is used to automatically format all the Rust kernel code,
including the generated C bindings (for details, please see
{\hyperref[\detokenize{coding-guidelines::doc}]{\sphinxcrossref{\DUrole{doc}{Coding Guidelines}}}}).

If \sphinxcode{\sphinxupquote{rustup}} is being used, its \sphinxcode{\sphinxupquote{default}} profile already installs the tool,
thus nothing needs to be done. If another profile is being used, the component
can be installed manually:

\begin{sphinxVerbatim}[commandchars=\\\{\}]
rustup component add rustfmt
\end{sphinxVerbatim}

The standalone installers also come with \sphinxcode{\sphinxupquote{rustfmt}}.


\subsection{clippy}
\label{\detokenize{quick-start:clippy}}
\sphinxcode{\sphinxupquote{clippy}} is a Rust linter. Running it provides extra warnings for Rust code.
It can be run by passing \sphinxcode{\sphinxupquote{CLIPPY=1}} to \sphinxcode{\sphinxupquote{make}} (for details, please see
{\hyperref[\detokenize{general-information::doc}]{\sphinxcrossref{\DUrole{doc}{General Information}}}}).

If \sphinxcode{\sphinxupquote{rustup}} is being used, its \sphinxcode{\sphinxupquote{default}} profile already installs the tool,
thus nothing needs to be done. If another profile is being used, the component
can be installed manually:

\begin{sphinxVerbatim}[commandchars=\\\{\}]
rustup component add clippy
\end{sphinxVerbatim}

The standalone installers also come with \sphinxcode{\sphinxupquote{clippy}}.


\subsection{cargo}
\label{\detokenize{quick-start:cargo}}
\sphinxcode{\sphinxupquote{cargo}} is the Rust native build system. It is currently required to run
the tests since it is used to build a custom standard library that contains
the facilities provided by the custom \sphinxcode{\sphinxupquote{alloc}} in the kernel. The tests can
be run using the \sphinxcode{\sphinxupquote{rusttest}} Make target.

If \sphinxcode{\sphinxupquote{rustup}} is being used, all the profiles already install the tool,
thus nothing needs to be done.

The standalone installers also come with \sphinxcode{\sphinxupquote{cargo}}.


\subsection{rustdoc}
\label{\detokenize{quick-start:rustdoc}}
\sphinxcode{\sphinxupquote{rustdoc}} is the documentation tool for Rust. It generates pretty HTML
documentation for Rust code (for details, please see
{\hyperref[\detokenize{general-information::doc}]{\sphinxcrossref{\DUrole{doc}{General Information}}}}).

\sphinxcode{\sphinxupquote{rustdoc}} is also used to test the examples provided in documented Rust code
(called doctests or documentation tests). The \sphinxcode{\sphinxupquote{rusttest}} Make target uses
this feature.

If \sphinxcode{\sphinxupquote{rustup}} is being used, all the profiles already install the tool,
thus nothing needs to be done.

The standalone installers also come with \sphinxcode{\sphinxupquote{rustdoc}}.


\subsection{rust\sphinxhyphen{}analyzer}
\label{\detokenize{quick-start:rust-analyzer}}
The \sphinxhref{https://rust-analyzer.github.io/}{rust\sphinxhyphen{}analyzer} language server can
be used with many editors to enable syntax highlighting, completion, go to
definition, and other features.

\sphinxcode{\sphinxupquote{rust\sphinxhyphen{}analyzer}} needs a configuration file, \sphinxcode{\sphinxupquote{rust\sphinxhyphen{}project.json}}, which
can be generated by the \sphinxcode{\sphinxupquote{rust\sphinxhyphen{}analyzer}} Make target.


\section{Configuration}
\label{\detokenize{quick-start:configuration}}
\sphinxcode{\sphinxupquote{Rust support}} (\sphinxcode{\sphinxupquote{CONFIG\_RUST}}) needs to be enabled in the \sphinxcode{\sphinxupquote{General setup}}
menu. The option is only shown if a suitable Rust toolchain is found (see
above), as long as the other requirements are met. In turn, this will make
visible the rest of options that depend on Rust.

Afterwards, go to:

\begin{sphinxVerbatim}[commandchars=\\\{\}]
Kernel hacking
    \PYGZhy{}\PYGZgt{} Sample kernel code
        \PYGZhy{}\PYGZgt{} Rust samples
\end{sphinxVerbatim}

And enable some sample modules either as built\sphinxhyphen{}in or as loadable.


\section{Building}
\label{\detokenize{quick-start:building}}
Building a kernel with a complete LLVM toolchain is the best supported setup
at the moment. That is:

\begin{sphinxVerbatim}[commandchars=\\\{\}]
make LLVM=1
\end{sphinxVerbatim}

For architectures that do not support a full LLVM toolchain, use:

\begin{sphinxVerbatim}[commandchars=\\\{\}]
make CC=clang
\end{sphinxVerbatim}

Using GCC also works for some configurations, but it is very experimental at
the moment.


\section{Hacking}
\label{\detokenize{quick-start:hacking}}
To dive deeper, take a look at the source code of the samples
at \sphinxcode{\sphinxupquote{samples/rust/}}, the Rust support code under \sphinxcode{\sphinxupquote{rust/}} and
the \sphinxcode{\sphinxupquote{Rust hacking}} menu under \sphinxcode{\sphinxupquote{Kernel hacking}}.

If GDB/Binutils is used and Rust symbols are not getting demangled, the reason
is the toolchain does not support Rust\textquotesingle{}s new v0 mangling scheme yet.
There are a few ways out:
\begin{itemize}
\item {} 
Install a newer release (GDB \textgreater{}= 10.2, Binutils \textgreater{}= 2.36).

\item {} 
Some versions of GDB (e.g. vanilla GDB 10.1) are able to use
the pre\sphinxhyphen{}demangled names embedded in the debug info (\sphinxcode{\sphinxupquote{CONFIG\_DEBUG\_INFO}}).

\end{itemize}


\chapter{General Information}
\label{\detokenize{general-information:general-information}}\label{\detokenize{general-information::doc}}
This document contains useful information to know when working with
the Rust support in the kernel.


\section{Code documentation}
\label{\detokenize{general-information:code-documentation}}
Rust kernel code is documented using \sphinxcode{\sphinxupquote{rustdoc}}, its built\sphinxhyphen{}in documentation
generator.

The generated HTML docs include integrated search, linked items (e.g. types,
functions, constants), source code, etc. They may be read at (TODO: link when
in mainline and generated alongside the rest of the documentation):
\begin{quote}

\sphinxurl{http://kernel.org/}
\end{quote}

The docs can also be easily generated and read locally. This is quite fast
(same order as compiling the code itself) and no special tools or environment
are needed. This has the added advantage that they will be tailored to
the particular kernel configuration used. To generate them, use the \sphinxcode{\sphinxupquote{rustdoc}}
target with the same invocation used for compilation, e.g.:

\begin{sphinxVerbatim}[commandchars=\\\{\}]
make LLVM=1 rustdoc
\end{sphinxVerbatim}

To read the docs locally in your web browser, run e.g.:

\begin{sphinxVerbatim}[commandchars=\\\{\}]
xdg\PYGZhy{}open rust/doc/kernel/index.html
\end{sphinxVerbatim}

To learn about how to write the documentation, please see {\hyperref[\detokenize{coding-guidelines::doc}]{\sphinxcrossref{\DUrole{doc}{Coding Guidelines}}}}.


\section{Extra lints}
\label{\detokenize{general-information:extra-lints}}
While \sphinxcode{\sphinxupquote{rustc}} is a very helpful compiler, some extra lints and analyses are
available via \sphinxcode{\sphinxupquote{clippy}}, a Rust linter. To enable it, pass \sphinxcode{\sphinxupquote{CLIPPY=1}} to
the same invocation used for compilation, e.g.:

\begin{sphinxVerbatim}[commandchars=\\\{\}]
make LLVM=1 CLIPPY=1
\end{sphinxVerbatim}

Please note that Clippy may change code generation, thus it should not be
enabled while building a production kernel.


\section{Abstractions vs. bindings}
\label{\detokenize{general-information:abstractions-vs-bindings}}
Abstractions are Rust code wrapping kernel functionality from the C side.

In order to use functions and types from the C side, bindings are created.
Bindings are the declarations for Rust of those functions and types from
the C side.

For instance, one may write a \sphinxcode{\sphinxupquote{Mutex}} abstraction in Rust which wraps
a \sphinxcode{\sphinxupquote{struct mutex}} from the C side and calls its functions through the bindings.

Abstractions are not available for all the kernel internal APIs and concepts,
but it is intended that coverage is expanded as time goes on. "Leaf" modules
(e.g. drivers) should not use the C bindings directly. Instead, subsystems
should provide as\sphinxhyphen{}safe\sphinxhyphen{}as\sphinxhyphen{}possible abstractions as needed.


\section{Conditional compilation}
\label{\detokenize{general-information:conditional-compilation}}
Rust code has access to conditional compilation based on the kernel
configuration:

\begin{sphinxVerbatim}[commandchars=\\\{\}]
\PYG{c+cp}{\PYGZsh{}[}\PYG{c+cp}{cfg(CONFIG\PYGZus{}X)}\PYG{c+cp}{]}\PYG{+w}{       }\PYG{c+c1}{// Enabled               (`y` or `m`)}
\PYG{c+cp}{\PYGZsh{}[}\PYG{c+cp}{cfg(CONFIG\PYGZus{}X=}\PYG{l+s}{\PYGZdq{}}\PYG{l+s}{y}\PYG{l+s}{\PYGZdq{}}\PYG{c+cp}{)}\PYG{c+cp}{]}\PYG{+w}{   }\PYG{c+c1}{// Enabled as a built\PYGZhy{}in (`y`)}
\PYG{c+cp}{\PYGZsh{}[}\PYG{c+cp}{cfg(CONFIG\PYGZus{}X=}\PYG{l+s}{\PYGZdq{}}\PYG{l+s}{m}\PYG{l+s}{\PYGZdq{}}\PYG{c+cp}{)}\PYG{c+cp}{]}\PYG{+w}{   }\PYG{c+c1}{// Enabled as a module   (`m`)}
\PYG{c+cp}{\PYGZsh{}[}\PYG{c+cp}{cfg(not(CONFIG\PYGZus{}X))}\PYG{c+cp}{]}\PYG{+w}{  }\PYG{c+c1}{// Disabled}
\end{sphinxVerbatim}


\chapter{Coding Guidelines}
\label{\detokenize{coding-guidelines:coding-guidelines}}\label{\detokenize{coding-guidelines::doc}}
This document describes how to write Rust code in the kernel.


\section{Style \& formatting}
\label{\detokenize{coding-guidelines:style-formatting}}
The code should be formatted using \sphinxcode{\sphinxupquote{rustfmt}}. In this way, a person
contributing from time to time to the kernel does not need to learn and
remember one more style guide. More importantly, reviewers and maintainers
do not need to spend time pointing out style issues anymore, and thus
less patch roundtrips may be needed to land a change.

\begin{sphinxadmonition}{note}{Note:}
Conventions on comments and documentation are not checked by
\sphinxcode{\sphinxupquote{rustfmt}}. Thus those are still needed to be taken care of.
\end{sphinxadmonition}

The default settings of \sphinxcode{\sphinxupquote{rustfmt}} are used. This means the idiomatic Rust
style is followed. For instance, 4 spaces are used for indentation rather
than tabs.

It is convenient to instruct editors/IDEs to format while typing,
when saving or at commit time. However, if for some reason reformatting
the entire kernel Rust sources is needed at some point, the following can be
run:

\begin{sphinxVerbatim}[commandchars=\\\{\}]
make LLVM=1 rustfmt
\end{sphinxVerbatim}

It is also possible to check if everything is formatted (printing a diff
otherwise), for instance for a CI, with:

\begin{sphinxVerbatim}[commandchars=\\\{\}]
make LLVM=1 rustfmtcheck
\end{sphinxVerbatim}

Like \sphinxcode{\sphinxupquote{clang\sphinxhyphen{}format}} for the rest of the kernel, \sphinxcode{\sphinxupquote{rustfmt}} works on
individual files, and does not require a kernel configuration. Sometimes it may
even work with broken code.


\section{Comments}
\label{\detokenize{coding-guidelines:comments}}
"Normal" comments (i.e. \sphinxcode{\sphinxupquote{//}}, rather than code documentation which starts
with \sphinxcode{\sphinxupquote{///}} or \sphinxcode{\sphinxupquote{//!}}) are written in Markdown the same way as documentation
comments are, even though they will not be rendered. This improves consistency,
simplifies the rules and allows to move content between the two kinds of
comments more easily. For instance:

\begin{sphinxVerbatim}[commandchars=\\\{\}]
\PYG{c+c1}{// `object` is ready to be handled now.}
\PYG{n}{f}\PYG{p}{(}\PYG{n}{object}\PYG{p}{)}\PYG{p}{;}
\end{sphinxVerbatim}

Furthermore, just like documentation, comments are capitalized at the beginning
of a sentence and ended with a period (even if it is a single sentence). This
includes \sphinxcode{\sphinxupquote{// SAFETY:}}, \sphinxcode{\sphinxupquote{// TODO:}} and other "tagged" comments, e.g.:

\begin{sphinxVerbatim}[commandchars=\\\{\}]
\PYG{c+c1}{// FIXME: The error should be handled properly.}
\end{sphinxVerbatim}

Comments should not be used for documentation purposes: comments are intended
for implementation details, not users. This distinction is useful even if the
reader of the source file is both an implementor and a user of an API. In fact,
sometimes it is useful to use both comments and documentation at the same time.
For instance, for a \sphinxcode{\sphinxupquote{TODO}} list or to comment on the documentation itself.
For the latter case, comments can be inserted in the middle; that is, closer to
the line of documentation to be commented. For any other case, comments are
written after the documentation, e.g.:

\begin{sphinxVerbatim}[commandchars=\\\{\}]
\PYG{l+s+sd}{/// Returns a new [`Foo`].}
\PYG{l+s+sd}{///}
\PYG{l+s+sd}{/// \PYGZsh{} Examples}
\PYG{l+s+sd}{///}
\PYG{c+c1}{// TODO: Find a better example.}
\PYG{l+s+sd}{/// ```}
\PYG{l+s+sd}{/// let foo = f(42);}
\PYG{l+s+sd}{/// ```}
\PYG{c+c1}{// FIXME: Use fallible approach.}
\PYG{k}{pub}\PYG{+w}{ }\PYG{k}{fn} \PYG{n+nf}{f}\PYG{p}{(}\PYG{n}{x}: \PYG{k+kt}{i32}\PYG{p}{)}\PYG{+w}{ }\PYGZhy{}\PYGZgt{} \PYG{n+nc}{Foo}\PYG{+w}{ }\PYG{p}{\PYGZob{}}
\PYG{+w}{    }\PYG{c+c1}{// ...}
\PYG{p}{\PYGZcb{}}
\end{sphinxVerbatim}

One special kind of comments are the \sphinxcode{\sphinxupquote{// SAFETY:}} comments. These must appear
before every \sphinxcode{\sphinxupquote{unsafe}} block, and they explain why the code inside the block is
correct/sound, i.e. why it cannot trigger undefined behavior in any case, e.g.:

\begin{sphinxVerbatim}[commandchars=\\\{\}]
\PYG{c+c1}{// SAFETY: `p` is valid by the safety requirements.}
\PYG{k}{unsafe}\PYG{+w}{ }\PYG{p}{\PYGZob{}}\PYG{+w}{ }\PYG{o}{*}\PYG{n}{p}\PYG{+w}{ }\PYG{o}{=}\PYG{+w}{ }\PYG{l+m+mi}{0}\PYG{p}{;}\PYG{+w}{ }\PYG{p}{\PYGZcb{}}
\end{sphinxVerbatim}

\sphinxcode{\sphinxupquote{// SAFETY:}} comments are not to be confused with the \sphinxcode{\sphinxupquote{\# Safety}} sections
in code documentation. \sphinxcode{\sphinxupquote{\# Safety}} sections specify the contract that callers
(for functions) or implementors (for traits) need to abide by. \sphinxcode{\sphinxupquote{// SAFETY:}}
comments show why a call (for functions) or implementation (for traits) actually
respects the preconditions stated in a \sphinxcode{\sphinxupquote{\# Safety}} section or the language
reference.


\section{Code documentation}
\label{\detokenize{coding-guidelines:code-documentation}}
Rust kernel code is not documented like C kernel code (i.e. via kernel\sphinxhyphen{}doc).
Instead, the usual system for documenting Rust code is used: the \sphinxcode{\sphinxupquote{rustdoc}}
tool, which uses Markdown (a lightweight markup language).

To learn Markdown, there are many guides available out there. For instance,
the one at:
\begin{quote}

\sphinxurl{https://commonmark.org/help/}
\end{quote}

This is how a well\sphinxhyphen{}documented Rust function may look like:

\begin{sphinxVerbatim}[commandchars=\\\{\}]
\PYG{l+s+sd}{/// Returns the contained [`Some`] value, consuming the `self` value,}
\PYG{l+s+sd}{/// without checking that the value is not [`None`].}
\PYG{l+s+sd}{///}
\PYG{l+s+sd}{/// \PYGZsh{} Safety}
\PYG{l+s+sd}{///}
\PYG{l+s+sd}{/// Calling this method on [`None`] is *[undefined behavior]*.}
\PYG{l+s+sd}{///}
\PYG{l+s+sd}{/// [undefined behavior]: https://doc.rust\PYGZhy{}lang.org/reference/behavior\PYGZhy{}considered\PYGZhy{}undefined.html}
\PYG{l+s+sd}{///}
\PYG{l+s+sd}{/// \PYGZsh{} Examples}
\PYG{l+s+sd}{///}
\PYG{l+s+sd}{/// ```}
\PYG{l+s+sd}{/// let x = Some(\PYGZdq{}air\PYGZdq{});}
\PYG{l+s+sd}{/// assert\PYGZus{}eq!(unsafe \PYGZob{} x.unwrap\PYGZus{}unchecked() \PYGZcb{}, \PYGZdq{}air\PYGZdq{});}
\PYG{l+s+sd}{/// ```}
\PYG{k}{pub}\PYG{+w}{ }\PYG{k}{unsafe}\PYG{+w}{ }\PYG{k}{fn} \PYG{n+nf}{unwrap\PYGZus{}unchecked}\PYG{p}{(}\PYG{n+nb+bp}{self}\PYG{p}{)}\PYG{+w}{ }\PYGZhy{}\PYGZgt{} \PYG{n+nc}{T}\PYG{+w}{ }\PYG{p}{\PYGZob{}}
\PYG{+w}{    }\PYG{k}{match}\PYG{+w}{ }\PYG{n+nb+bp}{self}\PYG{+w}{ }\PYG{p}{\PYGZob{}}
\PYG{+w}{        }\PYG{n+nb}{Some}\PYG{p}{(}\PYG{n}{val}\PYG{p}{)}\PYG{+w}{ }\PYG{o}{=}\PYG{o}{\PYGZgt{}}\PYG{+w}{ }\PYG{n}{val}\PYG{p}{,}

\PYG{+w}{        }\PYG{c+c1}{// SAFETY: The safety contract must be upheld by the caller.}
\PYG{+w}{        }\PYG{n+nb}{None}\PYG{+w}{ }\PYG{o}{=}\PYG{o}{\PYGZgt{}}\PYG{+w}{ }\PYG{k}{unsafe}\PYG{+w}{ }\PYG{p}{\PYGZob{}}\PYG{+w}{ }\PYG{n}{hint}::\PYG{n}{unreachable\PYGZus{}unchecked}\PYG{p}{(}\PYG{p}{)}\PYG{+w}{ }\PYG{p}{\PYGZcb{}}\PYG{p}{,}
\PYG{+w}{    }\PYG{p}{\PYGZcb{}}
\PYG{p}{\PYGZcb{}}
\end{sphinxVerbatim}

This example showcases a few \sphinxcode{\sphinxupquote{rustdoc}} features and some conventions followed
in the kernel:
\begin{itemize}
\item {} 
The first paragraph must be a single sentence briefly describing what
the documented item does. Further explanations must go in extra paragraphs.

\item {} 
Unsafe functions must document their safety preconditions under
a \sphinxcode{\sphinxupquote{\# Safety}} section.

\item {} 
While not shown here, if a function may panic, the conditions under which
that happens must be described under a \sphinxcode{\sphinxupquote{\# Panics}} section.

Please note that panicking should be very rare and used only with a good
reason. In almost all cases, a fallible approach should be used, typically
returning a \sphinxcode{\sphinxupquote{Result}}.

\item {} 
If providing examples of usage would help readers, they must be written in
a section called \sphinxcode{\sphinxupquote{\# Examples}}.

\item {} 
Rust items (functions, types, constants...) must be linked appropriately
(\sphinxcode{\sphinxupquote{rustdoc}} will create a link automatically).

\item {} 
Any \sphinxcode{\sphinxupquote{unsafe}} block must be preceded by a \sphinxcode{\sphinxupquote{// SAFETY:}} comment
describing why the code inside is sound.

While sometimes the reason might look trivial and therefore unneeded,
writing these comments is not just a good way of documenting what has been
taken into account, but most importantly, it provides a way to know that
there are no \sphinxstyleemphasis{extra} implicit constraints.

\end{itemize}

To learn more about how to write documentation for Rust and extra features,
please take a look at the \sphinxcode{\sphinxupquote{rustdoc}} book at:
\begin{quote}

\sphinxurl{https://doc.rust-lang.org/rustdoc/how-to-write-documentation.html}
\end{quote}


\section{Naming}
\label{\detokenize{coding-guidelines:naming}}
Rust kernel code follows the usual Rust naming conventions:
\begin{quote}

\sphinxurl{https://rust-lang.github.io/api-guidelines/naming.html}
\end{quote}

When existing C concepts (e.g. macros, functions, objects...) are wrapped into
a Rust abstraction, a name as close as reasonably possible to the C side should
be used in order to avoid confusion and to improve readability when switching
back and forth between the C and Rust sides. For instance, macros such as
\sphinxcode{\sphinxupquote{pr\_info}} from C are named the same in the Rust side.

Having said that, casing should be adjusted to follow the Rust naming
conventions, and namespacing introduced by modules and types should not be
repeated in the item names. For instance, when wrapping constants like:

\begin{sphinxVerbatim}[commandchars=\\\{\}]
\PYG{c+cp}{\PYGZsh{}}\PYG{c+cp}{define GPIO\PYGZus{}LINE\PYGZus{}DIRECTION\PYGZus{}IN  0}
\PYG{c+cp}{\PYGZsh{}}\PYG{c+cp}{define GPIO\PYGZus{}LINE\PYGZus{}DIRECTION\PYGZus{}OUT 1}
\end{sphinxVerbatim}

The equivalent in Rust may look like (ignoring documentation):

\begin{sphinxVerbatim}[commandchars=\\\{\}]
\PYG{k}{pub}\PYG{+w}{ }\PYG{k}{mod} \PYG{n+nn}{gpio}\PYG{+w}{ }\PYG{p}{\PYGZob{}}
\PYG{+w}{    }\PYG{k}{pub}\PYG{+w}{ }\PYG{k}{enum} \PYG{n+nc}{LineDirection}\PYG{+w}{ }\PYG{p}{\PYGZob{}}
\PYG{+w}{        }\PYG{n}{In}\PYG{+w}{ }\PYG{o}{=}\PYG{+w}{ }\PYG{n}{bindings}::\PYG{n}{GPIO\PYGZus{}LINE\PYGZus{}DIRECTION\PYGZus{}IN}\PYG{+w}{ }\PYG{k}{as}\PYG{+w}{ }\PYG{n}{\PYGZus{}}\PYG{p}{,}
\PYG{+w}{        }\PYG{n}{Out}\PYG{+w}{ }\PYG{o}{=}\PYG{+w}{ }\PYG{n}{bindings}::\PYG{n}{GPIO\PYGZus{}LINE\PYGZus{}DIRECTION\PYGZus{}OUT}\PYG{+w}{ }\PYG{k}{as}\PYG{+w}{ }\PYG{n}{\PYGZus{}}\PYG{p}{,}
\PYG{+w}{    }\PYG{p}{\PYGZcb{}}
\PYG{p}{\PYGZcb{}}
\end{sphinxVerbatim}

That is, the equivalent of \sphinxcode{\sphinxupquote{GPIO\_LINE\_DIRECTION\_IN}} would be referred to as
\sphinxcode{\sphinxupquote{gpio::LineDirection::In}}. In particular, it should not be named
\sphinxcode{\sphinxupquote{gpio::gpio\_line\_direction::GPIO\_LINE\_DIRECTION\_IN}}.


\chapter{Arch Support}
\label{\detokenize{arch-support:arch-support}}\label{\detokenize{arch-support::doc}}
Currently, the Rust compiler (\sphinxcode{\sphinxupquote{rustc}}) uses LLVM for code generation,
which limits the supported architectures that can be targeted. In addition,
support for building the kernel with LLVM/Clang varies (please see
Documentation/kbuild/llvm.rst). This support is needed for \sphinxcode{\sphinxupquote{bindgen}}
which uses \sphinxcode{\sphinxupquote{libclang}}.

Below is a general summary of architectures that currently work. Level of
support corresponds to \sphinxcode{\sphinxupquote{S}} values in the \sphinxcode{\sphinxupquote{MAINTAINERS}} file.


\begin{savenotes}\sphinxattablestart
\centering
\begin{tabulary}{\linewidth}[t]{|T|T|T|}
\hline
\sphinxstyletheadfamily 
Architecture
&\sphinxstyletheadfamily 
Level of support
&\sphinxstyletheadfamily 
Constraints
\\
\hline
\sphinxcode{\sphinxupquote{um}}
&
Maintained
&
\sphinxcode{\sphinxupquote{x86\_64}} only.
\\
\hline
\sphinxcode{\sphinxupquote{x86}}
&
Maintained
&
\sphinxcode{\sphinxupquote{x86\_64}} only.
\\
\hline
\end{tabulary}
\par
\sphinxattableend\end{savenotes}



\renewcommand{\indexname}{Index}
\printindex
\end{document}